\chapter{OpenMP: Scheduling, Reductions, Tasks, and Performance}
\label{chap:openmp}

\section{Introduction}
This lecture concludes our discussion of OpenMP. We have covered the basics in detail, and now focus on advanced constructs: scheduling strategies, reductions, task-based parallelism, and performance optimization. While OpenMP provides abstractions, fine-grained control is limited, and programmers must experiment to achieve optimal results.

\section{Scheduling Strategies}
OpenMP allows programmers to specify how loop iterations are distributed among threads.

\subsection{Static Scheduling}
\begin{itemize}
  \item Iterations are divided into fixed-size chunks before execution.
  \item Each chunk is assigned to a thread in round-robin fashion.
  \item Example: with $n$ threads, chunk $i$ goes to thread $(i \bmod n)$.
  \item Best suited for uniform workloads.
\end{itemize}

\subsection{Dynamic Scheduling}
\begin{itemize}
  \item Iterations are assigned at runtime as threads request work.
  \item Threads receive a chunk, complete it, then request another.
  \item Useful when workload per iteration is irregular.
\end{itemize}

\subsection{Guided Scheduling}
\begin{itemize}
  \item Starts with large chunks, then decreases chunk size as execution progresses.
  \item Provides adaptive load balancing.
  \item Left as a reading exercise for experimentation.
\end{itemize}

\section{Reductions}
Reductions combine partial results from multiple threads into a single result.

\subsection{Motivation}
\begin{itemize}
  \item Each thread computes a local partial result.
  \item These must be combined into a global result.
  \item Without reductions, explicit synchronization (atomic/critical) is required.
\end{itemize}

\subsection{Syntax}
\begin{lstlisting}[language=C, caption={Reduction over sum}]
#pragma omp parallel for reduction(+:sum)
for (int i = 0; i < N; i++) {
    sum += A[i];
}
\end{lstlisting}

\subsection{Supported Operators}
\begin{itemize}
  \item Arithmetic: +, *, -, /
  \item Logical: &&, ||
  \item Bitwise: &, |, ^
\end{itemize}

\subsection{Advantages}
\begin{itemize}
  \item Automatic handling of synchronization.
  \item Cleaner code compared to manual atomic/critical sections.
\end{itemize}

\section{Task-Based Parallelism and Sections}
OpenMP supports both data-parallel and task-parallel constructs.

\subsection{Tasks}
\begin{itemize}
  \item Represent independent units of work with dependencies.
  \item Enable MIMD-style parallelism.
\end{itemize}

\subsection{Sections}
\begin{lstlisting}[language=C, caption={Sections construct}]
#pragma omp parallel sections
{
  #pragma omp section
  computeA();

  #pragma omp section
  computeB();

  #pragma omp section
  computeC();
}
\end{lstlisting}

\begin{itemize}
  \item Each section is executed once, typically by one thread.
  \item Implicit barrier at the end ensures synchronization.
  \item Useful for dividing heterogeneous tasks among threads.
\end{itemize}

\section{Performance Considerations}
Parallelism introduces overheads. Programmers must balance workload and thread management.

\subsection{Overheads}
\begin{itemize}
  \item Thread creation and destruction.
  \item Context switching.
  \item Synchronization barriers.
\end{itemize}

\subsection{Guidelines}
\begin{itemize}
  \item Ensure sufficient iteration space (10,000–200,000 iterations) to mask overheads.
  \item Choose scheduling strategy based on workload uniformity.
  \item Experiment with chunk sizes and thread counts.
\end{itemize}

\subsection{Optimizations}
\begin{itemize}
  \item \textbf{SIMD Vectorization}: Use \verb|#pragma omp simd| to exploit intra-core parallelism.
  \item \textbf{Loop Collapsing}: Combine nested loops into a single loop to increase iteration space.
  \item \textbf{Loop Unrolling}: Reduce loop overhead by expanding iterations.
  \item \textbf{Avoid Locks}: Locks in loops add significant overhead.
\end{itemize}

\section{Conclusion}
OpenMP provides powerful abstractions for parallel programming. By understanding scheduling strategies, reductions, tasks, and performance trade-offs, programmers can design efficient parallel applications. 
